S\+DK \href{https://github.com/yangfuyuan/sdk/tree/non-singleton}{\tt test} application for Y\+D\+L\+I\+D\+AR

Visit E\+AI Website for more details about \href{http://www.ydlidar.com/}{\tt Y\+D\+L\+I\+D\+AR} .

\section*{How to build Y\+D\+L\+I\+D\+AR S\+DK samples }

\$ git clone \href{https://github.com/yangfuyuan/sdk}{\tt https\+://github.\+com/yangfuyuan/sdk} \$ cd sdk \$ git checkout non-\/singleton \$ cd .. \$ mkdir build \$ cd build \$ cmake ../sdk \$ make \#\#\#linux \$ vs open Project.\+sln \#\#\#windows

\section*{How to run Y\+D\+L\+I\+D\+AR S\+DK samples }

\$ cd samples

linux\+: \begin{DoxyVerb}$ ./ydlidar_test
$Please enter the lidar port:/dev/ttyUSB0
$Please enter the lidar baud rate:230400
\end{DoxyVerb}


windows\+: \begin{DoxyVerb}$ ydlidar_test.exe
$Please enter the lidar port:COM3
$Please enter the lidar baud rate:230400
\end{DoxyVerb}


===================================================================== You should see Y\+D\+L\+I\+D\+AR\textquotesingle{}s scan result in the console\+: \begin{DoxyVerb}Yd Lidar running correctly ! The health status: good
[YDLIDAR] Connection established in [/dev/ttyUSB0]:
Firmware version: 2.0.9
Hardware version: 2
Model: G4
Serial: 2018022700000003
[YDLIDAR INFO] Current Sampling Rate : 9K
[YDLIDAR INFO] Current Scan Frequency : 7.400000Hz
[YDLIDAR INFO] Now YDLIDAR is scanning ......
Scan received: 43 ranges
Scan received: 1361 ranges
Scan received: 1412 ranges
\end{DoxyVerb}


\section*{Upgrade Log }

2018-\/05-\/14 version\+:1.\+3.\+3

1.\+add the heart function constraint.

2.\+add packet type with scan frequency support.

2018-\/04-\/16 version\+:1.\+3.\+2

1.\+add multithreading support.

2018-\/04-\/16 version\+:1.\+3.\+1

1.\+Compensate for each laser point timestamp. 