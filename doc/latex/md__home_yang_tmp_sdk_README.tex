S\+DK \href{https://github.com/yangfuyuan/sdk}{\tt test} application for Y\+D\+L\+I\+D\+AR

Visit E\+AI Website for more details about \href{http://www.ydlidar.com/}{\tt Y\+D\+L\+I\+D\+AR} .

\section*{How to build Y\+D\+L\+I\+D\+AR S\+DK samples }

\$ git clone \href{https://github.com/yangfuyuan/sdk}{\tt https\+://github.\+com/yangfuyuan/sdk} \$ cd sdk \$ git checkout master \$ cd .. \$ mkdir build \$ cd build \$ cmake ../sdk \$ make \#\#\#linux \$ vs open Project.\+sln \#\#\#windows

\section*{How to run Y\+D\+L\+I\+D\+AR S\+DK samples }

\$ cd samples

linux\+: \begin{DoxyVerb}$ ./ydlidar_test
$Please enter the lidar port:/dev/ttyUSB0
$Please enter the lidar baud rate:230400
\end{DoxyVerb}


windows\+: \begin{DoxyVerb}$ ydlidar_test.exe
$Please enter the lidar port:COM3
$Please enter the lidar baud rate:230400
\end{DoxyVerb}


=====================================================================

You should see Y\+D\+L\+I\+D\+AR\textquotesingle{}s scan result in the console\+: \begin{DoxyVerb}Yd Lidar running correctly ! The health status: good
[YDLIDAR] Connection established in [/dev/ttyUSB0]:
Firmware version: 2.0.9
Hardware version: 2
Model: G4
Serial: 2018022700000003
[YDLIDAR INFO] Current Sampling Rate : 9K
[YDLIDAR INFO] Current Scan Frequency : 7.400000Hz
[YDLIDAR INFO] Now YDLIDAR is scanning ......
Scan received: 43 ranges
Scan received: 1361 ranges
Scan received: 1412 ranges
\end{DoxyVerb}


\section*{Lidar point data structure }

data structure\+: \begin{DoxyVerb}struct node_info {

   uint8_t    sync_quality;//!intensity

   uint16_t   angle_q6_checkbit; //!angle

   uint16_t   distance_q2; //! distance

   uint64_t   stamp; //! time stamp

   uint8_t    scan_frequence;//! current_frequence = scan_frequence/10.0, If the current value equals zero, it is an invalid value

} __attribute__((packed)) ;
\end{DoxyVerb}


example\+: \begin{DoxyVerb}if(data[i].scan_frequence != 0) {

    current_frequence = data[i].scan_frequence/10.0;
}

current_time_stamp = data[i].stamp;

current_distance = data[i].distance_q2/4.f;

current_angle = ((data[i].angle_q6_checkbit>>LIDAR_RESP_MEASUREMENT_ANGLE_SHIFT)/64.0f);

current_intensity = (float)(data[i].sync_quality >> 2);

###note:current_frequence = data[0].scan_frequence/10.0.

###if the current_frequence value equals zero, it is an invalid value.
\end{DoxyVerb}


code\+: \begin{DoxyVerb}    void ParseScan(node_info* data, const size_t& size) {

        double current_frequence, current_distance, current_angle, current_intensity;

        uint64_t current_time_stamp;

        for (size_t i = 0; i < size; i++ ) {

            if( data[i].scan_frequence != 0) {

                current_frequence =  data[i].scan_frequence;//or current_frequence = data[0].scan_frequence

            }

            current_time_stamp = data[i].stamp;

            current_angle = ((data[i].angle_q6_checkbit>>LIDAR_RESP_MEASUREMENT_ANGLE_SHIFT)/64.0f);//LIDAR_RESP_MEASUREMENT_ANGLE_SHIFT equals 8

            current_distance =  data[i].distance_q2/4.f;

            current_intensity = (float)(data[i].sync_quality >> 2);

        }

        if (current_frequence != 0 ) {

            printf("current lidar scan frequency: %f\n", current_frequence);

        } else {

            printf("Current lidar does not support return scan frequency\n");

        }
    }
\end{DoxyVerb}


\section*{Upgrade Log }

2018-\/05-\/23 version\+:1.\+3.\+4

1.\+add automatic reconnection if there is an exception

2.\+add serial file lock.

2018-\/05-\/14 version\+:1.\+3.\+3

1.\+add the heart function constraint.

2.\+add packet type with scan frequency support.

2018-\/04-\/16 version\+:1.\+3.\+2

1.\+add multithreading support.

2018-\/04-\/16 version\+:1.\+3.\+1

1.\+Compensate for each laser point timestamp. 